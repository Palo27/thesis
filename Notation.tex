\addcontentsline{toc}{chapter}{\protect\numberline{}Notation}
\chapter*{Notation}

\begin{table}[ht]
%\label{znacenie}
%\caption{Zna�enie}
$\bm{v}(X)$\\
$\circ$\\
male o, namely o(1)\\

\centering
\begin{tabular}{|l|l|}
\hline
Symbol & Popis\\
\hline
%$\mathbb{R}$		& mno�ina re�lnych ��sel\\
%$\mathbb{N}$ (resp. $\mathbb{N}_{0}$) & mno�ina prirodzen�ch ��sel (resp. mno�ina prirodzen�ch ��sel s nulou )\\
$\mathcal{L}(X)$					& rozdelenie n�hodnej veli�iny $X$\\
$\mathcal{S}_{n}$					& mno�ina permut�ci� ��sel ${1,\dots,n}$\\
$J_{n} $									& J�nossyho miera\\
$j_{n}$										& hustota J�nossyho miery vo�i Lebesgueovej miere\\
$S_{n}$										& podmienen� funkcia pre�itia\\
$h_{n}$										& funkcia hazardu\\
$\mathcal{B}(\mathbb{R}^{n})$ & Borelovsk� podmno�iny $\mathbb{R}^{n}$ \\
$1:n$ 										& diskr�tny interval, mno�ina $\{1,2,\dots,n\}$\\
$u\in \mathbb{R}^{+}$ 	  & �as pozorovanej udalosti\\
$T$ 											& doba trvania pokusu\\
$N(t)$										& po�et udalost� $u$ v intervale $[0,t]$\\
$\bm{N}_{0:t}=\{u,u<t\}$  & vektor pozorovan�ch udalost� do �asu $t$\\
$\Delta$ 									& ve�kos� delenia intervalu $[0,T]$\\
$ N_{k}=I_{\left[u\in[(k-1)\Delta,\ k\Delta ]\right]}$ 				& v�skyt udalosti v intrevale $[(k-1)\Delta,k\Delta ]$\\
$\bm{N}_{1:k}=\{N_{1},\dots,N_{k}\}$ 	& vektor n�l a jednotiek pod�a v�skytov udalost�\\
$\bm{\psi}_{k}$ & hodnota vektoru parametrov v �ase $k\Delta$\\
$\lambda_{k}$ 	& podmienen� intenzita bodov�ho procesu v �ase $k\Delta$\\ %=\lambda(k\,\Delta|\psi_{k},N_{1:k-1})
$\bm{F}$				& evolu�n� matica\\
$\bm{\eta}_{k}$ & Gaussov �um\\
\hline
\end{tabular}
\end{table}